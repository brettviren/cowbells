\documentclass[]{beamer}
\usepackage{graphicx}
\setbeamertemplate{navigation symbols}{}
\useoutertheme{infolines}
\usecolortheme[named=violet]{structure}
\setbeamertemplate{items}[circle]
\title[dedx]{Check dE/dx in Cowbells Simulation}
\author{Brett Viren}
\date{\today}

\definecolor{rootpink}{RGB}{255,0,255}
\begin{document}

\maketitle


\begin{frame}
  \frametitle{Procedure}
  \begin{itemize}
  \item Check $\Delta E$ deposited vs $\Delta x$ step for all steps
    that start/stop inside a material.
  \item Check as a function of step size.
  \item Compare to PSTAR numbers.
  \end{itemize}

  Check materials:
  \begin{center}
    \begin{tabular}[h]{|r|l|}
      Material & Density (g/cc) \\
      \hline
      Water (sample) & 1.0 \\
      WBLS (sample) & 0.99 \\
      Teflon (wall) & 2.2 \\
    \end{tabular}
  \end{center}

\end{frame}

\begin{frame}[fragile]
  \frametitle{dE/dx deposited in water}
  \begin{columns}
    \begin{column}{0.5\paperwidth}
      \includegraphics[width=\textwidth,page=9,type=pdf,ext=.pdf,read=.pdf]{dedx_multiplot_Water_step0.01}

      \begin{center}
        \texttt{defaultCutValue} = 0.01 mm
      \end{center}
    \end{column}
    \begin{column}{0.5\paperwidth}
      \includegraphics[width=\textwidth,page=9,type=pdf,ext=.pdf,read=.pdf]{dedx_multiplot_Water_step1.0}      

      \begin{center}
        \texttt{defaultCutValue} = 1.0 mm
      \end{center}
    \end{column}
  \end{columns}
  
  \begin{center}
    Line is PSTAR total stopping power.
  \end{center}
\end{frame}

\begin{frame}[fragile]
  \frametitle{dE/dx deposited in WBLS}
  \begin{columns}
    \begin{column}{0.5\paperwidth}
      \includegraphics[width=\textwidth,page=9,type=pdf,ext=.pdf,read=.pdf]{dedx_multiplot_WBLS_step0.01}

      \begin{center}
        \texttt{defaultCutValue} = 0.01 mm
      \end{center}
    \end{column}
    \begin{column}{0.5\paperwidth}
      \includegraphics[width=\textwidth,page=9,type=pdf,ext=.pdf,read=.pdf]{dedx_multiplot_WBLS_step1.0}      

      \begin{center}
        \texttt{defaultCutValue} = 1.0 mm
      \end{center}
    \end{column}
  \end{columns}
  
  \begin{center}
    Line is PSTAR total stopping power.
  \end{center}
\end{frame}

\begin{frame}[fragile]
  \frametitle{dE/dx deposited in Teflon}
  \begin{columns}
    \begin{column}{0.5\paperwidth}
      \includegraphics[width=\textwidth,page=9,type=pdf,ext=.pdf,read=.pdf]{dedx_multiplot_Teflon_step0.01}

      \begin{center}
        \texttt{defaultCutValue} = 0.01 mm
      \end{center}
    \end{column}
    \begin{column}{0.5\paperwidth}
      \includegraphics[width=\textwidth,page=9,type=pdf,ext=.pdf,read=.pdf]{dedx_multiplot_Teflon_step1.0}      

      \begin{center}
        \texttt{defaultCutValue} = 1.0 mm
      \end{center}
    \end{column}
  \end{columns}
  
  \begin{center}
    Line is PSTAR total stopping power.
  \end{center}
\end{frame}


\begin{frame}[fragile]
  \frametitle{Problems found and fixed}
  
  WbLS dE/dx higher than water.
  \begin{itemize}
  \item \texttt{WCSIM\_WbLS} code apparently had WbLS entered with
    element count fractions instead of mass fractions.
  \item[$\rightarrow$] fix: use corrected numbers from Minfang.
    \begin{center}\footnotesize
    \begin{tabular}[h]{rlcl}
Hydrogen & 0.659    & $\rightarrow$ & 0.1097 \\
Oxygen   & 0.309    & $\rightarrow$ & 0.8234 \\
Sulfur   & 0.0009   & $\rightarrow$ & 0.0048 \\
Nitrogen & 0.000058 & $\rightarrow$ & 0.0001 \\
Carbon   & 0.031    & $\rightarrow$ & 0.0620 \\
    \end{tabular}
    \end{center}
  \end{itemize}
\end{frame}

\begin{frame}[fragile]
  \frametitle{Open Issues}

    \begin{itemize}
    \item Why is dE/dx for water lower than PSTAR
      \begin{itemize}
      \item[$\rightarrow$] check with muons
      \end{itemize}
    \end{itemize}
  
\end{frame}

\end{document}