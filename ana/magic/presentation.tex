\documentclass[xcolor=dvipsnames]{beamer}

\input{beamer-preamble}

\begin{document}


\title{Simulated 2 GeV Protons in Water and WBLS01 in ``magic box'' detector}
\author{Brett Viren}
\date{\today}
\frame{\titlepage}

\begin{frame}
  \frametitle{Contents}
  \tableofcontents
\end{frame}


\section{Running the Simulation}
\label{sec:gen}

\begin{frame}[fragile]
  \frametitle{Running the Simulation}

  Generate the configuration

{\tiny
\begin{verbatim}
gennsrl.py 13a-water 13a-water.json sample=Water
gennsrl.py 13a-wbls01 13a-wbls01.json sample=WBLS01
\end{verbatim}
}

  Run G4

{\tiny
\begin{verbatim}
cowbells.exe -m hits,steps -n 1000 -p em,op \
  -k 'kin://beam?vertex=0,0,-5100&name=proton&direction=0,0,1&energy=2000' \
  -o nsrl-13a-2gev-protons-water.hits-steps.1k.root 13a-water.json 
cowbells.exe -m hits,steps -n 10 -p em,op \
  -k 'kin://beam?vertex=0,0,-5100&name=proton&direction=0,0,1&energy=2000' \
  -o nsrl-13a-2gev-protons-wbls01.hits-steps.1k.root 13a-wbls01.json
\end{verbatim}
}

 Results:
{\tiny
\begin{verbatim}
$ ls -sh
645M nsrl-13a-2gev-protons-water.hits-steps.1k.root
891M nsrl-13a-2gev-protons-wbls01.hits-steps.1k.root
\end{verbatim}
}

Plots (code in \texttt{cowbells/ana/magic}):
  \begin{lstlisting}[lang=python]
import steps
steps.xrayplot('water') # event display
steps.dedxplot('water') # per setp
import magic
magic.plots()           # per process
  \end{lstlisting}

\end{frame}


\section{Reflectivity}

\begin{frame}
  \frametitle{Reflectivity in magic box}

  Mean number of hits in up and downstream PMTs for 2GeV protons
  through water measured with 1000 protons.

\begin{center}
\begin{tabular}{rrrr}
\hline
refl\% & upstream & downstream & u/d (\%)\\
\hline
2 & 0.186 & 34.8 & 0.53\\
5 & 0.454 & 36.3 & 1.25\\
10 & 0.312 & 38.3 & 0.81\\
25 & 2.64 & 45.3 & 5.83\\
50 & 14.1 & 57.3 & 24.61\\
100 & 143 & 129 & 110.85\\
\hline
\end{tabular}
\end{center}

Rerun with 10k events for 0, 2, 5 and 10\%

\begin{center}
\begin{tabular}{rrrrrr}
\hline
refl\% & upstream & downstream & u/d  (\%) & u/d unc (\%) & \# up\\
\hline
0 & 2143 & 337947 & 0.63 & 0.014 & 280\\
2 & 1263 & 346834 & 0.36 & 0.010 & 464\\
5 & 1770 & 358916 & 0.49 & 0.012 & 949\\
10 & 4023 & 383227 & 1.05 & 0.017 & 2590\\
\hline
\end{tabular}
\end{center}

\end{frame}

\begin{frame}[fragile]
\frametitle{5\% reflectivity}
\includegraphics[width=0.49\textwidth]{images/refl/13a-water-ref0_05-us.pdf}
\includegraphics[width=0.49\textwidth]{images/refl/13a-water-ref0_05-ds.pdf}
\end{frame}
\begin{frame}[fragile]
\frametitle{50\% reflectivity}
\includegraphics[width=0.49\textwidth]{images/refl/13a-water-ref0_50-us.pdf}
\includegraphics[width=0.49\textwidth]{images/refl/13a-water-ref0_50-ds.pdf}
\end{frame}
\begin{frame}[fragile]
\frametitle{100\% reflectivity}
\includegraphics[width=0.49\textwidth]{images/refl/13a-water-ref1_00-us.pdf}
\includegraphics[width=0.49\textwidth]{images/refl/13a-water-ref1_00-ds.pdf}
\end{frame}

\section{Some Event Displays}

\begin{frame}[fragile]
  \frametitle{Event displays}
  \begin{itemize}
  \item First point of all steps in Water and WBLS from single or
    composite of multiple events:
  \item Layout is like:
  \end{itemize}

  \begin{center}
  \begin{tabular}[h]{|c|c|}
    \hline
    Z vs X & \\
    \hline
    Z vs Y & X vs Y \\
    \hline
  \end{tabular}
  \end{center}

\end{frame}

\frame{\includegraphics[width=\textwidth]{images/steps/xray-water-0.png}}
\frame{\includegraphics[width=\textwidth]{images/steps/xray-water-1.png}}
\frame{\includegraphics[width=\textwidth]{images/steps/xray-water-2.png}}
\frame{\includegraphics[width=\textwidth]{images/steps/xray-water-3.png}}
\frame{\includegraphics[width=\textwidth]{images/steps/xray-water-4.png}}
\frame{\includegraphics[width=\textwidth]{images/steps/xray-water-many.png}}

\frame{\includegraphics[width=\textwidth]{images/steps/xray-wbls01-0.png}}
\frame{\includegraphics[width=\textwidth]{images/steps/xray-wbls01-1.png}}
\frame{\includegraphics[width=\textwidth]{images/steps/xray-wbls01-2.png}}
\frame{\includegraphics[width=\textwidth]{images/steps/xray-wbls01-3.png}}
\frame{\includegraphics[width=\textwidth]{images/steps/xray-wbls01-4.png}}
\frame{\includegraphics[width=\textwidth]{images/steps/xray-wbls01-many.png}}

\section{Hits per Energy Lost}

\begin{frame}
  \frametitle{PE/MeV in Water and WBLS01}
  \begin{center}
    \includegraphics[width=0.95\textwidth]{images/steps/water-wbls01-hits-per-mev.pdf}    
  \end{center}
\end{frame}


\section{Timing and Charge by Process Type}

\begin{frame}
  \frametitle{Explanation of plots}
  Organized by 
  \begin{description}
  \item[Sample] \textbf{Water} vs. \textbf{WBLS01}, 2 GeV protons
  \item[Process] the physics process creating the particle that made
    the photon that made the hit.  Eg:
    \begin{description}
    \item[Cherenkov] a Cherenkov photon produced a different photon that hit the PMT
    \item[hioni] the hitting photon produced during hadron ionization (eg, primary proton)
    \item[primary] made up process for ``non process'' having to do with light directly made by the primary proton
    \end{description}
  \item[Channel] \textbf{PMT0} is downstream (facing beam), \textbf{PMT1} is upstream
  \item[Quantity] \textbf{timing} is every PMT hit time in all events in the category, \textbf{charge} is the total PE per event in the category.
  \end{description}
\end{frame}



\subsection{Timing}

\begin{frame}[fragile]
  \frametitle{Timing:  ``primary'' and ``hioni''(Water)}

\includegraphics[width=0.45\textwidth]{magic_plots/water_timing_primary_pmt0.pdf}%
\includegraphics[width=0.45\textwidth]{magic_plots/water_timing_primary_pmt1.pdf}%

\includegraphics[width=0.45\textwidth]{magic_plots/water_timing_hioni_pmt0.pdf}%
\includegraphics[width=0.45\textwidth]{magic_plots/water_timing_hioni_pmt1.pdf}%
\end{frame}

\begin{frame}[fragile]
  \frametitle{Timing: ``primary'' and ``hioni'' (WBLS01)}

\includegraphics[width=0.45\textwidth]{magic_plots/wbls01_timing_primary_pmt0.pdf}%
\includegraphics[width=0.45\textwidth]{magic_plots/wbls01_timing_primary_pmt1.pdf}%

\includegraphics[width=0.45\textwidth]{magic_plots/wbls01_timing_hioni_pmt0.pdf}%
\includegraphics[width=0.45\textwidth]{magic_plots/wbls01_timing_hioni_pmt1.pdf}%
\end{frame}


\subsection{Charge}

\begin{frame}[fragile]
  \frametitle{Charge: ``primary'' and ``hioni'' (Water)}

\includegraphics[width=0.45\textwidth]{magic_plots/water_charge_primary_pmt0.pdf}%
\includegraphics[width=0.45\textwidth]{magic_plots/water_charge_primary_pmt1.pdf}%

\includegraphics[width=0.45\textwidth]{magic_plots/water_charge_hioni_pmt0.pdf}%
\includegraphics[width=0.45\textwidth]{magic_plots/water_charge_hioni_pmt1.pdf}%
\end{frame}

\begin{frame}[fragile]
  \frametitle{Charge: ``primary'' and ``hioni'' (WBLS01)}

\includegraphics[width=0.45\textwidth]{magic_plots/wbls01_charge_primary_pmt0.pdf}%
\includegraphics[width=0.45\textwidth]{magic_plots/wbls01_charge_primary_pmt1.pdf}%

\includegraphics[width=0.45\textwidth]{magic_plots/wbls01_charge_hioni_pmt0.pdf}%
\includegraphics[width=0.45\textwidth]{magic_plots/wbls01_charge_hioni_pmt1.pdf}%
\end{frame}

\subsection {Whither Cherenkov}

\begin{frame}[fragile]
  \frametitle{Photons from Cherenkov Photons}
  (but not hits from \v{C}-photons themselves)

\includegraphics[width=0.42\textwidth]{magic_plots/wbls01_timing_cerenkov_pmt0.pdf}%
\includegraphics[width=0.42\textwidth]{magic_plots/wbls01_charge_cerenkov_pmt0.pdf}%

\includegraphics[width=0.42\textwidth]{magic_plots/wbls01_timing_cerenkov_pmt1.pdf}%
\includegraphics[width=0.42\textwidth]{magic_plots/wbls01_charge_cerenkov_pmt1.pdf}%

\end{frame}


\section{To do}

\begin{frame}
  \frametitle{To Do List}
  \begin{itemize}
  \item Add cosmic-$\mu$ spectrum as kinematics to cowbells
  \item Try to reproduce the Tubs' data for cosmics/protons
  \item ...
  \end{itemize}
\end{frame}

\end{document}
